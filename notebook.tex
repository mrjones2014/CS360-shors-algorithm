
% Default to the notebook output style

    


% Inherit from the specified cell style.




    
\documentclass[11pt]{article}

    
    
    \usepackage[T1]{fontenc}
    % Nicer default font (+ math font) than Computer Modern for most use cases
    \usepackage{mathpazo}

    % Basic figure setup, for now with no caption control since it's done
    % automatically by Pandoc (which extracts ![](path) syntax from Markdown).
    \usepackage{graphicx}
    % We will generate all images so they have a width \maxwidth. This means
    % that they will get their normal width if they fit onto the page, but
    % are scaled down if they would overflow the margins.
    \makeatletter
    \def\maxwidth{\ifdim\Gin@nat@width>\linewidth\linewidth
    \else\Gin@nat@width\fi}
    \makeatother
    \let\Oldincludegraphics\includegraphics
    % Set max figure width to be 80% of text width, for now hardcoded.
    \renewcommand{\includegraphics}[1]{\Oldincludegraphics[width=.8\maxwidth]{#1}}
    % Ensure that by default, figures have no caption (until we provide a
    % proper Figure object with a Caption API and a way to capture that
    % in the conversion process - todo).
    \usepackage{caption}
    \DeclareCaptionLabelFormat{nolabel}{}
    \captionsetup{labelformat=nolabel}

    \usepackage{adjustbox} % Used to constrain images to a maximum size 
    \usepackage{xcolor} % Allow colors to be defined
    \usepackage{enumerate} % Needed for markdown enumerations to work
    \usepackage{geometry} % Used to adjust the document margins
    \usepackage{amsmath} % Equations
    \usepackage{amssymb} % Equations
    \usepackage{textcomp} % defines textquotesingle
    % Hack from http://tex.stackexchange.com/a/47451/13684:
    \AtBeginDocument{%
        \def\PYZsq{\textquotesingle}% Upright quotes in Pygmentized code
    }
    \usepackage{upquote} % Upright quotes for verbatim code
    \usepackage{eurosym} % defines \euro
    \usepackage[mathletters]{ucs} % Extended unicode (utf-8) support
    \usepackage[utf8x]{inputenc} % Allow utf-8 characters in the tex document
    \usepackage{fancyvrb} % verbatim replacement that allows latex
    \usepackage{grffile} % extends the file name processing of package graphics 
                         % to support a larger range 
    % The hyperref package gives us a pdf with properly built
    % internal navigation ('pdf bookmarks' for the table of contents,
    % internal cross-reference links, web links for URLs, etc.)
    \usepackage{hyperref}
    \usepackage{longtable} % longtable support required by pandoc >1.10
    \usepackage{booktabs}  % table support for pandoc > 1.12.2
    \usepackage[inline]{enumitem} % IRkernel/repr support (it uses the enumerate* environment)
    \usepackage[normalem]{ulem} % ulem is needed to support strikethroughs (\sout)
                                % normalem makes italics be italics, not underlines
    

    
    
    % Colors for the hyperref package
    \definecolor{urlcolor}{rgb}{0,.145,.698}
    \definecolor{linkcolor}{rgb}{.71,0.21,0.01}
    \definecolor{citecolor}{rgb}{.12,.54,.11}

    % ANSI colors
    \definecolor{ansi-black}{HTML}{3E424D}
    \definecolor{ansi-black-intense}{HTML}{282C36}
    \definecolor{ansi-red}{HTML}{E75C58}
    \definecolor{ansi-red-intense}{HTML}{B22B31}
    \definecolor{ansi-green}{HTML}{00A250}
    \definecolor{ansi-green-intense}{HTML}{007427}
    \definecolor{ansi-yellow}{HTML}{DDB62B}
    \definecolor{ansi-yellow-intense}{HTML}{B27D12}
    \definecolor{ansi-blue}{HTML}{208FFB}
    \definecolor{ansi-blue-intense}{HTML}{0065CA}
    \definecolor{ansi-magenta}{HTML}{D160C4}
    \definecolor{ansi-magenta-intense}{HTML}{A03196}
    \definecolor{ansi-cyan}{HTML}{60C6C8}
    \definecolor{ansi-cyan-intense}{HTML}{258F8F}
    \definecolor{ansi-white}{HTML}{C5C1B4}
    \definecolor{ansi-white-intense}{HTML}{A1A6B2}

    % commands and environments needed by pandoc snippets
    % extracted from the output of `pandoc -s`
    \providecommand{\tightlist}{%
      \setlength{\itemsep}{0pt}\setlength{\parskip}{0pt}}
    \DefineVerbatimEnvironment{Highlighting}{Verbatim}{commandchars=\\\{\}}
    % Add ',fontsize=\small' for more characters per line
    \newenvironment{Shaded}{}{}
    \newcommand{\KeywordTok}[1]{\textcolor[rgb]{0.00,0.44,0.13}{\textbf{{#1}}}}
    \newcommand{\DataTypeTok}[1]{\textcolor[rgb]{0.56,0.13,0.00}{{#1}}}
    \newcommand{\DecValTok}[1]{\textcolor[rgb]{0.25,0.63,0.44}{{#1}}}
    \newcommand{\BaseNTok}[1]{\textcolor[rgb]{0.25,0.63,0.44}{{#1}}}
    \newcommand{\FloatTok}[1]{\textcolor[rgb]{0.25,0.63,0.44}{{#1}}}
    \newcommand{\CharTok}[1]{\textcolor[rgb]{0.25,0.44,0.63}{{#1}}}
    \newcommand{\StringTok}[1]{\textcolor[rgb]{0.25,0.44,0.63}{{#1}}}
    \newcommand{\CommentTok}[1]{\textcolor[rgb]{0.38,0.63,0.69}{\textit{{#1}}}}
    \newcommand{\OtherTok}[1]{\textcolor[rgb]{0.00,0.44,0.13}{{#1}}}
    \newcommand{\AlertTok}[1]{\textcolor[rgb]{1.00,0.00,0.00}{\textbf{{#1}}}}
    \newcommand{\FunctionTok}[1]{\textcolor[rgb]{0.02,0.16,0.49}{{#1}}}
    \newcommand{\RegionMarkerTok}[1]{{#1}}
    \newcommand{\ErrorTok}[1]{\textcolor[rgb]{1.00,0.00,0.00}{\textbf{{#1}}}}
    \newcommand{\NormalTok}[1]{{#1}}
    
    % Additional commands for more recent versions of Pandoc
    \newcommand{\ConstantTok}[1]{\textcolor[rgb]{0.53,0.00,0.00}{{#1}}}
    \newcommand{\SpecialCharTok}[1]{\textcolor[rgb]{0.25,0.44,0.63}{{#1}}}
    \newcommand{\VerbatimStringTok}[1]{\textcolor[rgb]{0.25,0.44,0.63}{{#1}}}
    \newcommand{\SpecialStringTok}[1]{\textcolor[rgb]{0.73,0.40,0.53}{{#1}}}
    \newcommand{\ImportTok}[1]{{#1}}
    \newcommand{\DocumentationTok}[1]{\textcolor[rgb]{0.73,0.13,0.13}{\textit{{#1}}}}
    \newcommand{\AnnotationTok}[1]{\textcolor[rgb]{0.38,0.63,0.69}{\textbf{\textit{{#1}}}}}
    \newcommand{\CommentVarTok}[1]{\textcolor[rgb]{0.38,0.63,0.69}{\textbf{\textit{{#1}}}}}
    \newcommand{\VariableTok}[1]{\textcolor[rgb]{0.10,0.09,0.49}{{#1}}}
    \newcommand{\ControlFlowTok}[1]{\textcolor[rgb]{0.00,0.44,0.13}{\textbf{{#1}}}}
    \newcommand{\OperatorTok}[1]{\textcolor[rgb]{0.40,0.40,0.40}{{#1}}}
    \newcommand{\BuiltInTok}[1]{{#1}}
    \newcommand{\ExtensionTok}[1]{{#1}}
    \newcommand{\PreprocessorTok}[1]{\textcolor[rgb]{0.74,0.48,0.00}{{#1}}}
    \newcommand{\AttributeTok}[1]{\textcolor[rgb]{0.49,0.56,0.16}{{#1}}}
    \newcommand{\InformationTok}[1]{\textcolor[rgb]{0.38,0.63,0.69}{\textbf{\textit{{#1}}}}}
    \newcommand{\WarningTok}[1]{\textcolor[rgb]{0.38,0.63,0.69}{\textbf{\textit{{#1}}}}}
    
    
    % Define a nice break command that doesn't care if a line doesn't already
    % exist.
    \def\br{\hspace*{\fill} \\* }
    % Math Jax compatability definitions
    \def\gt{>}
    \def\lt{<}
    % Document parameters
    \title{ShorsAlgorithm}
    
    
    

    % Pygments definitions
    
\makeatletter
\def\PY@reset{\let\PY@it=\relax \let\PY@bf=\relax%
    \let\PY@ul=\relax \let\PY@tc=\relax%
    \let\PY@bc=\relax \let\PY@ff=\relax}
\def\PY@tok#1{\csname PY@tok@#1\endcsname}
\def\PY@toks#1+{\ifx\relax#1\empty\else%
    \PY@tok{#1}\expandafter\PY@toks\fi}
\def\PY@do#1{\PY@bc{\PY@tc{\PY@ul{%
    \PY@it{\PY@bf{\PY@ff{#1}}}}}}}
\def\PY#1#2{\PY@reset\PY@toks#1+\relax+\PY@do{#2}}

\expandafter\def\csname PY@tok@w\endcsname{\def\PY@tc##1{\textcolor[rgb]{0.73,0.73,0.73}{##1}}}
\expandafter\def\csname PY@tok@c\endcsname{\let\PY@it=\textit\def\PY@tc##1{\textcolor[rgb]{0.25,0.50,0.50}{##1}}}
\expandafter\def\csname PY@tok@cp\endcsname{\def\PY@tc##1{\textcolor[rgb]{0.74,0.48,0.00}{##1}}}
\expandafter\def\csname PY@tok@k\endcsname{\let\PY@bf=\textbf\def\PY@tc##1{\textcolor[rgb]{0.00,0.50,0.00}{##1}}}
\expandafter\def\csname PY@tok@kp\endcsname{\def\PY@tc##1{\textcolor[rgb]{0.00,0.50,0.00}{##1}}}
\expandafter\def\csname PY@tok@kt\endcsname{\def\PY@tc##1{\textcolor[rgb]{0.69,0.00,0.25}{##1}}}
\expandafter\def\csname PY@tok@o\endcsname{\def\PY@tc##1{\textcolor[rgb]{0.40,0.40,0.40}{##1}}}
\expandafter\def\csname PY@tok@ow\endcsname{\let\PY@bf=\textbf\def\PY@tc##1{\textcolor[rgb]{0.67,0.13,1.00}{##1}}}
\expandafter\def\csname PY@tok@nb\endcsname{\def\PY@tc##1{\textcolor[rgb]{0.00,0.50,0.00}{##1}}}
\expandafter\def\csname PY@tok@nf\endcsname{\def\PY@tc##1{\textcolor[rgb]{0.00,0.00,1.00}{##1}}}
\expandafter\def\csname PY@tok@nc\endcsname{\let\PY@bf=\textbf\def\PY@tc##1{\textcolor[rgb]{0.00,0.00,1.00}{##1}}}
\expandafter\def\csname PY@tok@nn\endcsname{\let\PY@bf=\textbf\def\PY@tc##1{\textcolor[rgb]{0.00,0.00,1.00}{##1}}}
\expandafter\def\csname PY@tok@ne\endcsname{\let\PY@bf=\textbf\def\PY@tc##1{\textcolor[rgb]{0.82,0.25,0.23}{##1}}}
\expandafter\def\csname PY@tok@nv\endcsname{\def\PY@tc##1{\textcolor[rgb]{0.10,0.09,0.49}{##1}}}
\expandafter\def\csname PY@tok@no\endcsname{\def\PY@tc##1{\textcolor[rgb]{0.53,0.00,0.00}{##1}}}
\expandafter\def\csname PY@tok@nl\endcsname{\def\PY@tc##1{\textcolor[rgb]{0.63,0.63,0.00}{##1}}}
\expandafter\def\csname PY@tok@ni\endcsname{\let\PY@bf=\textbf\def\PY@tc##1{\textcolor[rgb]{0.60,0.60,0.60}{##1}}}
\expandafter\def\csname PY@tok@na\endcsname{\def\PY@tc##1{\textcolor[rgb]{0.49,0.56,0.16}{##1}}}
\expandafter\def\csname PY@tok@nt\endcsname{\let\PY@bf=\textbf\def\PY@tc##1{\textcolor[rgb]{0.00,0.50,0.00}{##1}}}
\expandafter\def\csname PY@tok@nd\endcsname{\def\PY@tc##1{\textcolor[rgb]{0.67,0.13,1.00}{##1}}}
\expandafter\def\csname PY@tok@s\endcsname{\def\PY@tc##1{\textcolor[rgb]{0.73,0.13,0.13}{##1}}}
\expandafter\def\csname PY@tok@sd\endcsname{\let\PY@it=\textit\def\PY@tc##1{\textcolor[rgb]{0.73,0.13,0.13}{##1}}}
\expandafter\def\csname PY@tok@si\endcsname{\let\PY@bf=\textbf\def\PY@tc##1{\textcolor[rgb]{0.73,0.40,0.53}{##1}}}
\expandafter\def\csname PY@tok@se\endcsname{\let\PY@bf=\textbf\def\PY@tc##1{\textcolor[rgb]{0.73,0.40,0.13}{##1}}}
\expandafter\def\csname PY@tok@sr\endcsname{\def\PY@tc##1{\textcolor[rgb]{0.73,0.40,0.53}{##1}}}
\expandafter\def\csname PY@tok@ss\endcsname{\def\PY@tc##1{\textcolor[rgb]{0.10,0.09,0.49}{##1}}}
\expandafter\def\csname PY@tok@sx\endcsname{\def\PY@tc##1{\textcolor[rgb]{0.00,0.50,0.00}{##1}}}
\expandafter\def\csname PY@tok@m\endcsname{\def\PY@tc##1{\textcolor[rgb]{0.40,0.40,0.40}{##1}}}
\expandafter\def\csname PY@tok@gh\endcsname{\let\PY@bf=\textbf\def\PY@tc##1{\textcolor[rgb]{0.00,0.00,0.50}{##1}}}
\expandafter\def\csname PY@tok@gu\endcsname{\let\PY@bf=\textbf\def\PY@tc##1{\textcolor[rgb]{0.50,0.00,0.50}{##1}}}
\expandafter\def\csname PY@tok@gd\endcsname{\def\PY@tc##1{\textcolor[rgb]{0.63,0.00,0.00}{##1}}}
\expandafter\def\csname PY@tok@gi\endcsname{\def\PY@tc##1{\textcolor[rgb]{0.00,0.63,0.00}{##1}}}
\expandafter\def\csname PY@tok@gr\endcsname{\def\PY@tc##1{\textcolor[rgb]{1.00,0.00,0.00}{##1}}}
\expandafter\def\csname PY@tok@ge\endcsname{\let\PY@it=\textit}
\expandafter\def\csname PY@tok@gs\endcsname{\let\PY@bf=\textbf}
\expandafter\def\csname PY@tok@gp\endcsname{\let\PY@bf=\textbf\def\PY@tc##1{\textcolor[rgb]{0.00,0.00,0.50}{##1}}}
\expandafter\def\csname PY@tok@go\endcsname{\def\PY@tc##1{\textcolor[rgb]{0.53,0.53,0.53}{##1}}}
\expandafter\def\csname PY@tok@gt\endcsname{\def\PY@tc##1{\textcolor[rgb]{0.00,0.27,0.87}{##1}}}
\expandafter\def\csname PY@tok@err\endcsname{\def\PY@bc##1{\setlength{\fboxsep}{0pt}\fcolorbox[rgb]{1.00,0.00,0.00}{1,1,1}{\strut ##1}}}
\expandafter\def\csname PY@tok@kc\endcsname{\let\PY@bf=\textbf\def\PY@tc##1{\textcolor[rgb]{0.00,0.50,0.00}{##1}}}
\expandafter\def\csname PY@tok@kd\endcsname{\let\PY@bf=\textbf\def\PY@tc##1{\textcolor[rgb]{0.00,0.50,0.00}{##1}}}
\expandafter\def\csname PY@tok@kn\endcsname{\let\PY@bf=\textbf\def\PY@tc##1{\textcolor[rgb]{0.00,0.50,0.00}{##1}}}
\expandafter\def\csname PY@tok@kr\endcsname{\let\PY@bf=\textbf\def\PY@tc##1{\textcolor[rgb]{0.00,0.50,0.00}{##1}}}
\expandafter\def\csname PY@tok@bp\endcsname{\def\PY@tc##1{\textcolor[rgb]{0.00,0.50,0.00}{##1}}}
\expandafter\def\csname PY@tok@fm\endcsname{\def\PY@tc##1{\textcolor[rgb]{0.00,0.00,1.00}{##1}}}
\expandafter\def\csname PY@tok@vc\endcsname{\def\PY@tc##1{\textcolor[rgb]{0.10,0.09,0.49}{##1}}}
\expandafter\def\csname PY@tok@vg\endcsname{\def\PY@tc##1{\textcolor[rgb]{0.10,0.09,0.49}{##1}}}
\expandafter\def\csname PY@tok@vi\endcsname{\def\PY@tc##1{\textcolor[rgb]{0.10,0.09,0.49}{##1}}}
\expandafter\def\csname PY@tok@vm\endcsname{\def\PY@tc##1{\textcolor[rgb]{0.10,0.09,0.49}{##1}}}
\expandafter\def\csname PY@tok@sa\endcsname{\def\PY@tc##1{\textcolor[rgb]{0.73,0.13,0.13}{##1}}}
\expandafter\def\csname PY@tok@sb\endcsname{\def\PY@tc##1{\textcolor[rgb]{0.73,0.13,0.13}{##1}}}
\expandafter\def\csname PY@tok@sc\endcsname{\def\PY@tc##1{\textcolor[rgb]{0.73,0.13,0.13}{##1}}}
\expandafter\def\csname PY@tok@dl\endcsname{\def\PY@tc##1{\textcolor[rgb]{0.73,0.13,0.13}{##1}}}
\expandafter\def\csname PY@tok@s2\endcsname{\def\PY@tc##1{\textcolor[rgb]{0.73,0.13,0.13}{##1}}}
\expandafter\def\csname PY@tok@sh\endcsname{\def\PY@tc##1{\textcolor[rgb]{0.73,0.13,0.13}{##1}}}
\expandafter\def\csname PY@tok@s1\endcsname{\def\PY@tc##1{\textcolor[rgb]{0.73,0.13,0.13}{##1}}}
\expandafter\def\csname PY@tok@mb\endcsname{\def\PY@tc##1{\textcolor[rgb]{0.40,0.40,0.40}{##1}}}
\expandafter\def\csname PY@tok@mf\endcsname{\def\PY@tc##1{\textcolor[rgb]{0.40,0.40,0.40}{##1}}}
\expandafter\def\csname PY@tok@mh\endcsname{\def\PY@tc##1{\textcolor[rgb]{0.40,0.40,0.40}{##1}}}
\expandafter\def\csname PY@tok@mi\endcsname{\def\PY@tc##1{\textcolor[rgb]{0.40,0.40,0.40}{##1}}}
\expandafter\def\csname PY@tok@il\endcsname{\def\PY@tc##1{\textcolor[rgb]{0.40,0.40,0.40}{##1}}}
\expandafter\def\csname PY@tok@mo\endcsname{\def\PY@tc##1{\textcolor[rgb]{0.40,0.40,0.40}{##1}}}
\expandafter\def\csname PY@tok@ch\endcsname{\let\PY@it=\textit\def\PY@tc##1{\textcolor[rgb]{0.25,0.50,0.50}{##1}}}
\expandafter\def\csname PY@tok@cm\endcsname{\let\PY@it=\textit\def\PY@tc##1{\textcolor[rgb]{0.25,0.50,0.50}{##1}}}
\expandafter\def\csname PY@tok@cpf\endcsname{\let\PY@it=\textit\def\PY@tc##1{\textcolor[rgb]{0.25,0.50,0.50}{##1}}}
\expandafter\def\csname PY@tok@c1\endcsname{\let\PY@it=\textit\def\PY@tc##1{\textcolor[rgb]{0.25,0.50,0.50}{##1}}}
\expandafter\def\csname PY@tok@cs\endcsname{\let\PY@it=\textit\def\PY@tc##1{\textcolor[rgb]{0.25,0.50,0.50}{##1}}}

\def\PYZbs{\char`\\}
\def\PYZus{\char`\_}
\def\PYZob{\char`\{}
\def\PYZcb{\char`\}}
\def\PYZca{\char`\^}
\def\PYZam{\char`\&}
\def\PYZlt{\char`\<}
\def\PYZgt{\char`\>}
\def\PYZsh{\char`\#}
\def\PYZpc{\char`\%}
\def\PYZdl{\char`\$}
\def\PYZhy{\char`\-}
\def\PYZsq{\char`\'}
\def\PYZdq{\char`\"}
\def\PYZti{\char`\~}
% for compatibility with earlier versions
\def\PYZat{@}
\def\PYZlb{[}
\def\PYZrb{]}
\makeatother


    % Exact colors from NB
    \definecolor{incolor}{rgb}{0.0, 0.0, 0.5}
    \definecolor{outcolor}{rgb}{0.545, 0.0, 0.0}



    
    % Prevent overflowing lines due to hard-to-break entities
    \sloppy 
    % Setup hyperref package
    \hypersetup{
      breaklinks=true,  % so long urls are correctly broken across lines
      colorlinks=true,
      urlcolor=urlcolor,
      linkcolor=linkcolor,
      citecolor=citecolor,
      }
    % Slightly bigger margins than the latex defaults
    
    \geometry{verbose,tmargin=1in,bmargin=1in,lmargin=1in,rmargin=1in}
    
    

    \begin{document}
    
    
    \maketitle
    
    

    
    \hypertarget{shors-algorithm}{%
\section{Shor's Algorithm}\label{shors-algorithm}}

\hypertarget{abstract}{%
\subsection{Abstract}\label{abstract}}

In 1995, Peter Shor, professor of applied mathematics at MIT, published
a polynomial-time quantum algorithm for prime factorization of large
(one thousand digits or more) integers. This is a problem thought to be
utterly infeasible on even the most powerful classical computers.

\hypertarget{implementation}{%
\subsection{Implementation}\label{implementation}}

\href{https://projectq.ch/}{ProjectQ} is an open-source Python 3
framework for running quantum algorithms on a quantum
simulator/emulator/IBM quantum chip (configurable via the `backend'
module imported).

Below is a Python ProjectQ implementation of Shor's algorithm, running
on the quantum simulator backend by default.

    \begin{Verbatim}[commandchars=\\\{\}]
{\color{incolor}In [{\color{incolor}4}]:} \PY{c+c1}{\PYZsh{} Sample taken from https://github.com/ProjectQ\PYZhy{}Framework/ProjectQ/blob/master/examples/shor.py}
        
        \PY{k+kn}{from} \PY{n+nn}{\PYZus{}\PYZus{}future\PYZus{}\PYZus{}} \PY{k}{import} \PY{n}{print\PYZus{}function}
        
        \PY{k+kn}{import} \PY{n+nn}{math}
        \PY{k+kn}{import} \PY{n+nn}{random}
        \PY{k+kn}{import} \PY{n+nn}{sys}
        \PY{k+kn}{from} \PY{n+nn}{fractions} \PY{k}{import} \PY{n}{Fraction}
        \PY{k}{try}\PY{p}{:}
            \PY{k+kn}{from} \PY{n+nn}{math} \PY{k}{import} \PY{n}{gcd}
        \PY{k}{except} \PY{n+ne}{ImportError}\PY{p}{:}
            \PY{k+kn}{from} \PY{n+nn}{fractions} \PY{k}{import} \PY{n}{gcd}
        
        \PY{k+kn}{from} \PY{n+nn}{builtins} \PY{k}{import} \PY{n+nb}{input}
        
        \PY{k+kn}{import} \PY{n+nn}{projectq}\PY{n+nn}{.}\PY{n+nn}{libs}\PY{n+nn}{.}\PY{n+nn}{math}
        \PY{k+kn}{import} \PY{n+nn}{projectq}\PY{n+nn}{.}\PY{n+nn}{setups}\PY{n+nn}{.}\PY{n+nn}{decompositions}
        \PY{k+kn}{from} \PY{n+nn}{projectq}\PY{n+nn}{.}\PY{n+nn}{backends} \PY{k}{import} \PY{n}{Simulator}\PY{p}{,} \PY{n}{IBMBackend}\PY{p}{,} \PY{n}{ResourceCounter}
        \PY{k+kn}{from} \PY{n+nn}{projectq}\PY{n+nn}{.}\PY{n+nn}{cengines} \PY{k}{import} \PY{p}{(}\PY{n}{MainEngine}\PY{p}{,}
                                       \PY{n}{AutoReplacer}\PY{p}{,}
                                       \PY{n}{LocalOptimizer}\PY{p}{,}
                                       \PY{n}{TagRemover}\PY{p}{,}
                                       \PY{n}{InstructionFilter}\PY{p}{,}
                                       \PY{n}{DecompositionRuleSet}\PY{p}{)}
        \PY{k+kn}{from} \PY{n+nn}{projectq}\PY{n+nn}{.}\PY{n+nn}{libs}\PY{n+nn}{.}\PY{n+nn}{math} \PY{k}{import} \PY{p}{(}\PY{n}{AddConstant}\PY{p}{,}
                                        \PY{n}{AddConstantModN}\PY{p}{,}
                                        \PY{n}{MultiplyByConstantModN}\PY{p}{)}
        \PY{k+kn}{from} \PY{n+nn}{projectq}\PY{n+nn}{.}\PY{n+nn}{meta} \PY{k}{import} \PY{n}{Control}
        \PY{k+kn}{from} \PY{n+nn}{projectq}\PY{n+nn}{.}\PY{n+nn}{ops} \PY{k}{import} \PY{p}{(}\PY{n}{X}\PY{p}{,}
                                  \PY{n}{Measure}\PY{p}{,}
                                  \PY{n}{H}\PY{p}{,}
                                  \PY{n}{R}\PY{p}{,}
                                  \PY{n}{QFT}\PY{p}{,}
                                  \PY{n}{Swap}\PY{p}{,}
                                  \PY{n}{get\PYZus{}inverse}\PY{p}{,}
                                  \PY{n}{BasicMathGate}\PY{p}{)}
        
        
        \PY{k}{def} \PY{n+nf}{run\PYZus{}shor}\PY{p}{(}\PY{n}{eng}\PY{p}{,} \PY{n}{N}\PY{p}{,} \PY{n}{a}\PY{p}{,} \PY{n}{verbose}\PY{o}{=}\PY{k+kc}{False}\PY{p}{)}\PY{p}{:}
            \PY{l+s+sd}{\PYZdq{}\PYZdq{}\PYZdq{}}
        \PY{l+s+sd}{    Runs the quantum subroutine of Shor\PYZsq{}s algorithm for factoring.}
        
        \PY{l+s+sd}{    Args:}
        \PY{l+s+sd}{        eng (MainEngine): Main compiler engine to use.}
        \PY{l+s+sd}{        N (int): Number to factor.}
        \PY{l+s+sd}{        a (int): Relative prime to use as a base for a\PYZca{}x mod N.}
        \PY{l+s+sd}{        verbose (bool): If True, display intermediate measurement results.}
        
        \PY{l+s+sd}{    Returns:}
        \PY{l+s+sd}{        r (float): Potential period of a.}
        \PY{l+s+sd}{    \PYZdq{}\PYZdq{}\PYZdq{}}
            \PY{n}{n} \PY{o}{=} \PY{n+nb}{int}\PY{p}{(}\PY{n}{math}\PY{o}{.}\PY{n}{ceil}\PY{p}{(}\PY{n}{math}\PY{o}{.}\PY{n}{log}\PY{p}{(}\PY{n}{N}\PY{p}{,} \PY{l+m+mi}{2}\PY{p}{)}\PY{p}{)}\PY{p}{)}
        
            \PY{n}{x} \PY{o}{=} \PY{n}{eng}\PY{o}{.}\PY{n}{allocate\PYZus{}qureg}\PY{p}{(}\PY{n}{n}\PY{p}{)}
        
            \PY{n}{X} \PY{o}{|} \PY{n}{x}\PY{p}{[}\PY{l+m+mi}{0}\PY{p}{]}
        
            \PY{n}{measurements} \PY{o}{=} \PY{p}{[}\PY{l+m+mi}{0}\PY{p}{]} \PY{o}{*} \PY{p}{(}\PY{l+m+mi}{2} \PY{o}{*} \PY{n}{n}\PY{p}{)}  \PY{c+c1}{\PYZsh{} will hold the 2n measurement results}
        
            \PY{n}{ctrl\PYZus{}qubit} \PY{o}{=} \PY{n}{eng}\PY{o}{.}\PY{n}{allocate\PYZus{}qubit}\PY{p}{(}\PY{p}{)}
        
            \PY{k}{for} \PY{n}{k} \PY{o+ow}{in} \PY{n+nb}{range}\PY{p}{(}\PY{l+m+mi}{2} \PY{o}{*} \PY{n}{n}\PY{p}{)}\PY{p}{:}
                \PY{n}{current\PYZus{}a} \PY{o}{=} \PY{n+nb}{pow}\PY{p}{(}\PY{n}{a}\PY{p}{,} \PY{l+m+mi}{1} \PY{o}{\PYZlt{}\PYZlt{}} \PY{p}{(}\PY{l+m+mi}{2} \PY{o}{*} \PY{n}{n} \PY{o}{\PYZhy{}} \PY{l+m+mi}{1} \PY{o}{\PYZhy{}} \PY{n}{k}\PY{p}{)}\PY{p}{,} \PY{n}{N}\PY{p}{)}
                \PY{c+c1}{\PYZsh{} one iteration of 1\PYZhy{}qubit QPE}
                \PY{n}{H} \PY{o}{|} \PY{n}{ctrl\PYZus{}qubit}
                \PY{k}{with} \PY{n}{Control}\PY{p}{(}\PY{n}{eng}\PY{p}{,} \PY{n}{ctrl\PYZus{}qubit}\PY{p}{)}\PY{p}{:}
                    \PY{n}{MultiplyByConstantModN}\PY{p}{(}\PY{n}{current\PYZus{}a}\PY{p}{,} \PY{n}{N}\PY{p}{)} \PY{o}{|} \PY{n}{x}
        
                \PY{c+c1}{\PYZsh{} perform inverse QFT \PYZhy{}\PYZhy{}\PYZgt{} Rotations conditioned on previous outcomes}
                \PY{k}{for} \PY{n}{i} \PY{o+ow}{in} \PY{n+nb}{range}\PY{p}{(}\PY{n}{k}\PY{p}{)}\PY{p}{:}
                    \PY{k}{if} \PY{n}{measurements}\PY{p}{[}\PY{n}{i}\PY{p}{]}\PY{p}{:}
                        \PY{n}{R}\PY{p}{(}\PY{o}{\PYZhy{}}\PY{n}{math}\PY{o}{.}\PY{n}{pi}\PY{o}{/}\PY{p}{(}\PY{l+m+mi}{1} \PY{o}{\PYZlt{}\PYZlt{}} \PY{p}{(}\PY{n}{k} \PY{o}{\PYZhy{}} \PY{n}{i}\PY{p}{)}\PY{p}{)}\PY{p}{)} \PY{o}{|} \PY{n}{ctrl\PYZus{}qubit}
                \PY{n}{H} \PY{o}{|} \PY{n}{ctrl\PYZus{}qubit}
        
                \PY{c+c1}{\PYZsh{} and measure}
                \PY{n}{Measure} \PY{o}{|} \PY{n}{ctrl\PYZus{}qubit}
                \PY{n}{eng}\PY{o}{.}\PY{n}{flush}\PY{p}{(}\PY{p}{)}
                \PY{n}{measurements}\PY{p}{[}\PY{n}{k}\PY{p}{]} \PY{o}{=} \PY{n+nb}{int}\PY{p}{(}\PY{n}{ctrl\PYZus{}qubit}\PY{p}{)}
                \PY{k}{if} \PY{n}{measurements}\PY{p}{[}\PY{n}{k}\PY{p}{]}\PY{p}{:}
                    \PY{n}{X} \PY{o}{|} \PY{n}{ctrl\PYZus{}qubit}
        
                \PY{k}{if} \PY{n}{verbose}\PY{p}{:}
                    \PY{n+nb}{print}\PY{p}{(}\PY{l+s+s2}{\PYZdq{}}\PY{l+s+se}{\PYZbs{}033}\PY{l+s+s2}{[95m}\PY{l+s+si}{\PYZob{}\PYZcb{}}\PY{l+s+se}{\PYZbs{}033}\PY{l+s+s2}{[0m}\PY{l+s+s2}{\PYZdq{}}\PY{o}{.}\PY{n}{format}\PY{p}{(}\PY{n}{measurements}\PY{p}{[}\PY{n}{k}\PY{p}{]}\PY{p}{)}\PY{p}{,} \PY{n}{end}\PY{o}{=}\PY{l+s+s2}{\PYZdq{}}\PY{l+s+s2}{\PYZdq{}}\PY{p}{)}
                    \PY{n}{sys}\PY{o}{.}\PY{n}{stdout}\PY{o}{.}\PY{n}{flush}\PY{p}{(}\PY{p}{)}
        
            \PY{n}{Measure} \PY{o}{|} \PY{n}{x}
            \PY{c+c1}{\PYZsh{} turn the measured values into a number in [0,1)}
            \PY{n}{y} \PY{o}{=} \PY{n+nb}{sum}\PY{p}{(}\PY{p}{[}\PY{p}{(}\PY{n}{measurements}\PY{p}{[}\PY{l+m+mi}{2} \PY{o}{*} \PY{n}{n} \PY{o}{\PYZhy{}} \PY{l+m+mi}{1} \PY{o}{\PYZhy{}} \PY{n}{i}\PY{p}{]}\PY{o}{*}\PY{l+m+mf}{1.} \PY{o}{/} \PY{p}{(}\PY{l+m+mi}{1} \PY{o}{\PYZlt{}\PYZlt{}} \PY{p}{(}\PY{n}{i} \PY{o}{+} \PY{l+m+mi}{1}\PY{p}{)}\PY{p}{)}\PY{p}{)}
                     \PY{k}{for} \PY{n}{i} \PY{o+ow}{in} \PY{n+nb}{range}\PY{p}{(}\PY{l+m+mi}{2} \PY{o}{*} \PY{n}{n}\PY{p}{)}\PY{p}{]}\PY{p}{)}
        
            \PY{c+c1}{\PYZsh{} continued fraction expansion to get denominator (the period?)}
            \PY{n}{r} \PY{o}{=} \PY{n}{Fraction}\PY{p}{(}\PY{n}{y}\PY{p}{)}\PY{o}{.}\PY{n}{limit\PYZus{}denominator}\PY{p}{(}\PY{n}{N}\PY{o}{\PYZhy{}}\PY{l+m+mi}{1}\PY{p}{)}\PY{o}{.}\PY{n}{denominator}
        
            \PY{c+c1}{\PYZsh{} return the (potential) period}
            \PY{k}{return} \PY{n}{r}
        
        
        \PY{c+c1}{\PYZsh{} Filter function, which defines the gate set for the first optimization}
        \PY{c+c1}{\PYZsh{} (don\PYZsq{}t decompose QFTs and iQFTs to make cancellation easier)}
        \PY{k}{def} \PY{n+nf}{high\PYZus{}level\PYZus{}gates}\PY{p}{(}\PY{n}{eng}\PY{p}{,} \PY{n}{cmd}\PY{p}{)}\PY{p}{:}
            \PY{n}{g} \PY{o}{=} \PY{n}{cmd}\PY{o}{.}\PY{n}{gate}
            \PY{k}{if} \PY{n}{g} \PY{o}{==} \PY{n}{QFT} \PY{o+ow}{or} \PY{n}{get\PYZus{}inverse}\PY{p}{(}\PY{n}{g}\PY{p}{)} \PY{o}{==} \PY{n}{QFT} \PY{o+ow}{or} \PY{n}{g} \PY{o}{==} \PY{n}{Swap}\PY{p}{:}
                \PY{k}{return} \PY{k+kc}{True}
            \PY{k}{if} \PY{n+nb}{isinstance}\PY{p}{(}\PY{n}{g}\PY{p}{,} \PY{n}{BasicMathGate}\PY{p}{)}\PY{p}{:}
                \PY{k}{return} \PY{k+kc}{False}
                \PY{k}{if} \PY{n+nb}{isinstance}\PY{p}{(}\PY{n}{g}\PY{p}{,} \PY{n}{AddConstant}\PY{p}{)}\PY{p}{:}
                    \PY{k}{return} \PY{k+kc}{True}
                \PY{k}{elif} \PY{n+nb}{isinstance}\PY{p}{(}\PY{n}{g}\PY{p}{,} \PY{n}{AddConstantModN}\PY{p}{)}\PY{p}{:}
                    \PY{k}{return} \PY{k+kc}{True}
                \PY{k}{return} \PY{k+kc}{False}
            \PY{k}{return} \PY{n}{eng}\PY{o}{.}\PY{n}{next\PYZus{}engine}\PY{o}{.}\PY{n}{is\PYZus{}available}\PY{p}{(}\PY{n}{cmd}\PY{p}{)}
        
        
        \PY{k}{if} \PY{n+nv+vm}{\PYZus{}\PYZus{}name\PYZus{}\PYZus{}} \PY{o}{==} \PY{l+s+s2}{\PYZdq{}}\PY{l+s+s2}{\PYZus{}\PYZus{}main\PYZus{}\PYZus{}}\PY{l+s+s2}{\PYZdq{}}\PY{p}{:}
            \PY{c+c1}{\PYZsh{} build compilation engine list}
            \PY{n}{resource\PYZus{}counter} \PY{o}{=} \PY{n}{ResourceCounter}\PY{p}{(}\PY{p}{)}
            \PY{n}{rule\PYZus{}set} \PY{o}{=} \PY{n}{DecompositionRuleSet}\PY{p}{(}\PY{n}{modules}\PY{o}{=}\PY{p}{[}\PY{n}{projectq}\PY{o}{.}\PY{n}{libs}\PY{o}{.}\PY{n}{math}\PY{p}{,}
                                                     \PY{n}{projectq}\PY{o}{.}\PY{n}{setups}\PY{o}{.}\PY{n}{decompositions}\PY{p}{]}\PY{p}{)}
            \PY{n}{compilerengines} \PY{o}{=} \PY{p}{[}\PY{n}{AutoReplacer}\PY{p}{(}\PY{n}{rule\PYZus{}set}\PY{p}{)}\PY{p}{,}
                               \PY{n}{InstructionFilter}\PY{p}{(}\PY{n}{high\PYZus{}level\PYZus{}gates}\PY{p}{)}\PY{p}{,}
                               \PY{n}{TagRemover}\PY{p}{(}\PY{p}{)}\PY{p}{,}
                               \PY{n}{LocalOptimizer}\PY{p}{(}\PY{l+m+mi}{3}\PY{p}{)}\PY{p}{,}
                               \PY{n}{AutoReplacer}\PY{p}{(}\PY{n}{rule\PYZus{}set}\PY{p}{)}\PY{p}{,}
                               \PY{n}{TagRemover}\PY{p}{(}\PY{p}{)}\PY{p}{,}
                               \PY{n}{LocalOptimizer}\PY{p}{(}\PY{l+m+mi}{3}\PY{p}{)}\PY{p}{,}
                               \PY{n}{resource\PYZus{}counter}\PY{p}{]}
        
            
            \PY{c+c1}{\PYZsh{} make the compiler and run the circuit on the simulator backend}
            \PY{n}{eng} \PY{o}{=} \PY{n}{MainEngine}\PY{p}{(}\PY{n}{Simulator}\PY{p}{(}\PY{p}{)}\PY{p}{,} \PY{n}{compilerengines}\PY{p}{)}
        
            \PY{c+c1}{\PYZsh{} print welcome message and ask the user for the number to factor}
            \PY{n+nb}{print}\PY{p}{(}\PY{l+s+s2}{\PYZdq{}}\PY{l+s+se}{\PYZbs{}n}\PY{l+s+se}{\PYZbs{}t}\PY{l+s+se}{\PYZbs{}033}\PY{l+s+s2}{[37mprojectq}\PY{l+s+se}{\PYZbs{}033}\PY{l+s+s2}{[0m}\PY{l+s+se}{\PYZbs{}n}\PY{l+s+se}{\PYZbs{}t}\PY{l+s+s2}{\PYZhy{}\PYZhy{}\PYZhy{}\PYZhy{}\PYZhy{}\PYZhy{}\PYZhy{}\PYZhy{}}\PY{l+s+se}{\PYZbs{}n}\PY{l+s+se}{\PYZbs{}t}\PY{l+s+s2}{Implementation of Shor}\PY{l+s+s2}{\PYZdq{}}
                  \PY{l+s+s2}{\PYZdq{}}\PY{l+s+se}{\PYZbs{}\PYZsq{}}\PY{l+s+s2}{s algorithm.}\PY{l+s+s2}{\PYZdq{}}\PY{p}{,} \PY{n}{end}\PY{o}{=}\PY{l+s+s2}{\PYZdq{}}\PY{l+s+s2}{\PYZdq{}}\PY{p}{)}
            \PY{n}{N} \PY{o}{=} \PY{n+nb}{int}\PY{p}{(}\PY{n+nb}{input}\PY{p}{(}\PY{l+s+s1}{\PYZsq{}}\PY{l+s+se}{\PYZbs{}n}\PY{l+s+se}{\PYZbs{}t}\PY{l+s+s1}{Number to factor: }\PY{l+s+s1}{\PYZsq{}}\PY{p}{)}\PY{p}{)}
            \PY{n+nb}{print}\PY{p}{(}\PY{l+s+s2}{\PYZdq{}}\PY{l+s+se}{\PYZbs{}n}\PY{l+s+se}{\PYZbs{}t}\PY{l+s+s2}{Factoring N = }\PY{l+s+si}{\PYZob{}\PYZcb{}}\PY{l+s+s2}{: }\PY{l+s+se}{\PYZbs{}033}\PY{l+s+s2}{[0m}\PY{l+s+s2}{\PYZdq{}}\PY{o}{.}\PY{n}{format}\PY{p}{(}\PY{n}{N}\PY{p}{)}\PY{p}{,} \PY{n}{end}\PY{o}{=}\PY{l+s+s2}{\PYZdq{}}\PY{l+s+s2}{\PYZdq{}}\PY{p}{)}
        
            \PY{c+c1}{\PYZsh{} choose a base at random:}
            \PY{n}{a} \PY{o}{=} \PY{n+nb}{int}\PY{p}{(}\PY{n}{random}\PY{o}{.}\PY{n}{random}\PY{p}{(}\PY{p}{)}\PY{o}{*}\PY{n}{N}\PY{p}{)}
            \PY{k}{if} \PY{o+ow}{not} \PY{n}{gcd}\PY{p}{(}\PY{n}{a}\PY{p}{,} \PY{n}{N}\PY{p}{)} \PY{o}{==} \PY{l+m+mi}{1}\PY{p}{:}
                \PY{n+nb}{print}\PY{p}{(}\PY{l+s+s2}{\PYZdq{}}\PY{l+s+se}{\PYZbs{}n}\PY{l+s+se}{\PYZbs{}n}\PY{l+s+se}{\PYZbs{}t}\PY{l+s+se}{\PYZbs{}033}\PY{l+s+s2}{[92mOoops, we were lucky: Chose non relative prime}\PY{l+s+s2}{\PYZdq{}}
                      \PY{l+s+s2}{\PYZdq{}}\PY{l+s+s2}{ by accident :)}\PY{l+s+s2}{\PYZdq{}}\PY{p}{)}
                \PY{n+nb}{print}\PY{p}{(}\PY{l+s+s2}{\PYZdq{}}\PY{l+s+se}{\PYZbs{}t}\PY{l+s+s2}{Factor: }\PY{l+s+si}{\PYZob{}\PYZcb{}}\PY{l+s+se}{\PYZbs{}033}\PY{l+s+s2}{[0m}\PY{l+s+s2}{\PYZdq{}}\PY{o}{.}\PY{n}{format}\PY{p}{(}\PY{n}{gcd}\PY{p}{(}\PY{n}{a}\PY{p}{,} \PY{n}{N}\PY{p}{)}\PY{p}{)}\PY{p}{)}
            \PY{k}{else}\PY{p}{:}
                \PY{c+c1}{\PYZsh{} run the quantum subroutine}
                \PY{n}{r} \PY{o}{=} \PY{n}{run\PYZus{}shor}\PY{p}{(}\PY{n}{eng}\PY{p}{,} \PY{n}{N}\PY{p}{,} \PY{n}{a}\PY{p}{,} \PY{k+kc}{True}\PY{p}{)}
        
                \PY{c+c1}{\PYZsh{} try to determine the factors}
                \PY{k}{if} \PY{n}{r} \PY{o}{\PYZpc{}} \PY{l+m+mi}{2} \PY{o}{!=} \PY{l+m+mi}{0}\PY{p}{:}
                    \PY{n}{r} \PY{o}{*}\PY{o}{=} \PY{l+m+mi}{2}
                \PY{n}{apowrhalf} \PY{o}{=} \PY{n+nb}{pow}\PY{p}{(}\PY{n}{a}\PY{p}{,} \PY{n}{r} \PY{o}{\PYZgt{}\PYZgt{}} \PY{l+m+mi}{1}\PY{p}{,} \PY{n}{N}\PY{p}{)}
                \PY{n}{f1} \PY{o}{=} \PY{n}{gcd}\PY{p}{(}\PY{n}{apowrhalf} \PY{o}{+} \PY{l+m+mi}{1}\PY{p}{,} \PY{n}{N}\PY{p}{)}
                \PY{n}{f2} \PY{o}{=} \PY{n}{gcd}\PY{p}{(}\PY{n}{apowrhalf} \PY{o}{\PYZhy{}} \PY{l+m+mi}{1}\PY{p}{,} \PY{n}{N}\PY{p}{)}
                \PY{k}{if} \PY{p}{(}\PY{p}{(}\PY{o+ow}{not} \PY{n}{f1} \PY{o}{*} \PY{n}{f2} \PY{o}{==} \PY{n}{N}\PY{p}{)} \PY{o+ow}{and} \PY{n}{f1} \PY{o}{*} \PY{n}{f2} \PY{o}{\PYZgt{}} \PY{l+m+mi}{1} \PY{o+ow}{and}
                        \PY{n+nb}{int}\PY{p}{(}\PY{l+m+mf}{1.} \PY{o}{*} \PY{n}{N} \PY{o}{/} \PY{p}{(}\PY{n}{f1} \PY{o}{*} \PY{n}{f2}\PY{p}{)}\PY{p}{)} \PY{o}{*} \PY{n}{f1} \PY{o}{*} \PY{n}{f2} \PY{o}{==} \PY{n}{N}\PY{p}{)}\PY{p}{:}
                    \PY{n}{f1}\PY{p}{,} \PY{n}{f2} \PY{o}{=} \PY{n}{f1}\PY{o}{*}\PY{n}{f2}\PY{p}{,} \PY{n+nb}{int}\PY{p}{(}\PY{n}{N}\PY{o}{/}\PY{p}{(}\PY{n}{f1}\PY{o}{*}\PY{n}{f2}\PY{p}{)}\PY{p}{)}
                \PY{k}{if} \PY{n}{f1} \PY{o}{*} \PY{n}{f2} \PY{o}{==} \PY{n}{N} \PY{o+ow}{and} \PY{n}{f1} \PY{o}{\PYZgt{}} \PY{l+m+mi}{1} \PY{o+ow}{and} \PY{n}{f2} \PY{o}{\PYZgt{}} \PY{l+m+mi}{1}\PY{p}{:}
                    \PY{n+nb}{print}\PY{p}{(}\PY{l+s+s2}{\PYZdq{}}\PY{l+s+se}{\PYZbs{}n}\PY{l+s+se}{\PYZbs{}n}\PY{l+s+se}{\PYZbs{}t}\PY{l+s+se}{\PYZbs{}033}\PY{l+s+s2}{[92mFactors found: }\PY{l+s+si}{\PYZob{}\PYZcb{}}\PY{l+s+s2}{ * }\PY{l+s+si}{\PYZob{}\PYZcb{}}\PY{l+s+s2}{ = }\PY{l+s+si}{\PYZob{}\PYZcb{}}\PY{l+s+se}{\PYZbs{}033}\PY{l+s+s2}{[0m}\PY{l+s+s2}{\PYZdq{}}
                          \PY{o}{.}\PY{n}{format}\PY{p}{(}\PY{n}{f1}\PY{p}{,} \PY{n}{f2}\PY{p}{,} \PY{n}{N}\PY{p}{)}\PY{p}{)}
                \PY{k}{else}\PY{p}{:}
                    \PY{n+nb}{print}\PY{p}{(}\PY{l+s+s2}{\PYZdq{}}\PY{l+s+se}{\PYZbs{}n}\PY{l+s+se}{\PYZbs{}n}\PY{l+s+se}{\PYZbs{}t}\PY{l+s+se}{\PYZbs{}033}\PY{l+s+s2}{[91mBad luck: Found }\PY{l+s+si}{\PYZob{}\PYZcb{}}\PY{l+s+s2}{ and }\PY{l+s+si}{\PYZob{}\PYZcb{}}\PY{l+s+se}{\PYZbs{}033}\PY{l+s+s2}{[0m}\PY{l+s+s2}{\PYZdq{}}\PY{o}{.}\PY{n}{format}\PY{p}{(}\PY{n}{f1}\PY{p}{,}
                                                                                  \PY{n}{f2}\PY{p}{)}\PY{p}{)}
        
                \PY{c+c1}{\PYZsh{} print(resource\PYZus{}counter)  \PYZsh{} print resource usage}
\end{Verbatim}


    \begin{Verbatim}[commandchars=\\\{\}]
(Note: This is the (slow) Python simulator.)

	\textcolor{ansi-white}{projectq}
	--------
	Implementation of Shor's algorithm.
	Number to factor: 15

	Factoring N = 15: \textcolor{ansi-magenta-intense}{0}\textcolor{ansi-magenta-intense}{0}\textcolor{ansi-magenta-intense}{0}\textcolor{ansi-magenta-intense}{0}\textcolor{ansi-magenta-intense}{0}\textcolor{ansi-magenta-intense}{0}\textcolor{ansi-magenta-intense}{1}\textcolor{ansi-magenta-intense}{0}

	\textcolor{ansi-green-intense}{Factors found: 5 * 3 = 15}

    \end{Verbatim}


    % Add a bibliography block to the postdoc
    
    
    
    \end{document}
